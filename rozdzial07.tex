\chapter{Podsumowanie}
\section{Wykonane prace}
	Stworzone zostało kompletne rozwiązanie pozwalające na sprawną rozbudowę zaawansowanej aplikacji.
	Zastosowano najnowsze rozwiązania z dziedziny projektowania stron internetowych,
	co pozwoliło na stworzenie uniwersalnej strony działającej zarówno na komputerach, jak i telefonach komórkowych.
	Jednocześnie dzięki zapytaniom wykorzystującym protokół GraphQL oraz zaawansowanemu systemowi cache, zużycie danych przesyłanych przez internet jest ograniczone do minimum.
	Dzięki temu GraphQL jest obecnie postrzegane jako przyszłość projektowania API dla stron internetowych typu SPA.

	Zaimplementowano jeden z najbezpieczniejszych na chwilę obecną systemów uwierzytelniania i autoryzacji, OpenID Connect.
	Wykorzystany tryb uwierzytelniania z użyciem kodu autoryzacji i PKCE zapewnia największą ochronę użytkownika korzystającego ze strony w przeglądarce internetowej.
	Podczas tworzenia widoków utrzymano konwencję \emph{Material design}, uzyskując nowoczesny wygląd zgodny ze standardem na urządzeniach z systemem Android.
	Całość aplikacji stworzona została na platformie Docker, dzięki czemu może zostać bezproblemowo uruchomiona na każdej maszynie. 
	
	Zgodnie z założeniami, w projekcie wykorzystano tylko rozwiązania open source.
	Kod stworzonej aplikacji również udostępniony jest publicznie na platformie GitHub.
	Wykorzystanie systemu kontroli wersji Git umożliwia łatwe rozbudowanie aplikacji oraz ułatwia współpracę,
	jeśli osoby trzecie chciałyby rozpocząć współpracę przy rozwijaniu programu.
	
\section{Wnioski}
	GraphQL pozwala na budowanie zaawansowanego API przy utrzymaniu zrozumiałej i czytelnej architektury API (brak powielanych endpoint-ów).
	Rozwiązuje dwa typowe problemy API: \emph{underfetching} oraz \emph{overfetching}, czyli odpowiednio pobieranie za mało lub za dużo danych z API.
	Wymaga jednak implementacji skomplikowanego cache i uniemożliwia stworzenie cache na poziomie sieci.
	Należy również pamiętać, że jest to jedynie protokół przesyłania danych i formatowania zapytań.
	Nie rozwiązuje problemów optymalizacyjnych występujących w API typu RESTful (np.\ problem N+1).
	W dalszym ciągu potrzebna jest dodatkowa logika, która optymalizować będzie zapytania do bazy.

	Serwisy stworzone w .NET Core sprawdzają się bardzo dobrze pod kątem bezpieczeństwa oraz stabilności.
	Stworzone przez Microsoft biblioteki pomocnicze pozwalają na stworzenie zaawansowanej funkcjonalności
	z zachowaniem pewności co do jakości kodu.

	
\section{Możliwości dalszej rozbudowy}
	Całość projektu stanowi zaawansowany szkielet, na podstawie którego można stworzyć rozbudowaną aplikację.

	Serwis API powinien zostać zoptymalizowany pod kątem zapytań do bazy.
	Dodatkowo, jeśli przewiduje się dużo danych, po stronie API należy zaimplementować mechanizm paginacji i sortowania.
	Zarówno Hot Chocolate, jak i Entity Framework Core wspierają operacje tego typu.

	Zastosowany framework Angular oferuje wsparcie dla PWA (ang.\ \emph{Progressive Web Application}) poprzez dodanie service worker-ów.
	Są to pomocnicze skrypty działające w tle, dzięki którym strona może działać bez połączenia z internetem.
	Aplikacja została zbudowana wykorzystując wiele procesów automatyzacji, dzięki czemu możliwe jest bardzo szybkie rozwijanie API o dodatkowe modele bazy i typy GraphQL.
	Użycie Entity Framework Core pozwala na proste modyfikacje nawet na istniejącej bazie poprzez zarządzanie migracjami.

