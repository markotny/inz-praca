% !TEX root = ./Dyplom.tex

\chapter{Przegląd technologii}
\label{sec:przegTech}

\section{Paradygmaty komunikacji z usługami sieciowymi}
	Powstało wiele sposobów komunikacji z usługami sieciowymi.
	W tym rozdziale opisane zostaną najpopularniejsze z nich.

	\subsection{RPC}
		RPC to prosty protokół zdalnego wywołania procedury.
		Klient wysyła do serwera potrzebne dane do wywołania akcji: nazwę metody oraz parametry.
		W podstawowej wersji całość komunikacji odbywa się za pomocą jednego punktu dostępowego (ang.\ \emph{endpoint}).
		Zwykle wykorzystywane są wariacje protokołu np. JSON-RPC, które przesyłają dane w formacie JSON.

		Przykładowe wywołanie oraz odpowiedź:
		\begin{figure}[!htb]
			\centering
			\begin{minipage}{.5\textwidth}
				\begin{lstlisting}[label=lst:rpcReq]
					{
						"jsonrpc": "2.0",
						"method": "divide",
						"params": {
							"dividend": 32,
							"divisor": 4
						},
						"id": 5
					}
				\end{lstlisting}	
			\end{minipage}%
			\begin{minipage}{0.5\textwidth}
				\begin{lstlisting}[label=lst:rpcRes]
					{
						"jsonrpc": "2.0",
						"result": 8,
						"id": 5
					}
				\end{lstlisting}
			\end{minipage}
		\end{figure}

	\subsection{SOAP}
		SOAP to ustandaryzowany protokół służący do opakowywania przesyłanych danych. 
		Często porównywany do wkładania danych do ,,koperty'', definiuje sposób kodowania i dekodowania danych do formatu XML~\cite{SOAP}.
		Paczka z danymi zawiera również informacje opisujące dane oraz w jaki sposób mają być przetworzone.
		Wykorzystuje przez to więcej zasobów niż czysty JSON z danymi -- zarówno w kwestii rozmiaru przesyłanych danych jak i czasu ich przetworzenia.
		
		Opcjonalny lecz zalecany jest dokument WSDL, który w ustandaryzowany sposób określa punkty dostępowe usług serwisu.
		Umożliwia automatyczne generowanie testów serwisu oraz funkcji wywołujących usługi.
		
		Zapytanie w SOAP jest zawsze typu POST, a więc odpowiedź nie jest zapisywana przez przeglądarkę i za każdym razem wywoływana jest procedura w Api.
		Jest jednak dzięki temu bardziej zaawansowany - zapewnia wsparcie dla transakcji spełniających ACID
		oraz posiada rozszerzone mechanizmy bezpieczeństwa -- oprócz SSL, implementuje również WS-Security.
		Te właściwości są istotnymi zaletami wykorzystywanymi w aplikacjach typu enterprise.\newline
		Przykładowa budowa zapytania SOAP~\cite{SoapMsg}:

		\begin{lstlisting}[label=lst:soapReq, xleftmargin=0.5cm]
			<?xml version="1.0"?>

			<soap:Envelope
			xmlns:soap="http://www.w3.org/2003/05/soap-envelope/"
			soap:encodingStyle="http://www.w3.org/2003/05/soap-encoding">

			<soap:Body xmlns:m="http://www.example.org/stock">
				<m:GetStockPrice>
					<m:StockName>IBM</m:StockName>
				</m:GetStockPrice>
			</soap:Body>

			</soap:Envelope>
		\end{lstlisting}
		oraz odpowiedź:
		\begin{lstlisting}[label=lst:soapRes, xleftmargin=0.5cm]
			<?xml version="1.0"?>

			<soap:Envelope
			xmlns:soap="http://www.w3.org/2003/05/soap-envelope/"
			soap:encodingStyle="http://www.w3.org/2003/05/soap-encoding">

			<soap:Body xmlns:m="http://www.example.org/stock">
				<m:GetStockPriceResponse>
					<m:Price>34.5</m:Price>
				</m:GetStockPriceResponse>
			</soap:Body>

			</soap:Envelope>
		\end{lstlisting}
			
	\subsection{REST}
		REST nie jest protokołem lecz stylem architektury.
		Nie narzuca konkretnej implementacji ani specyfikacji. Definiuje za to wymagania, które powinno spełnić Api typu RESTful~\cite{RESTful}:
		\begin{enumerate}
			\item Klient-Serwer: podział ról i odpowiedzialności -- klient i serwer mogą być rozwijane niezależnie, o ile interfejs pozostanie bez zmian.
			\item Bezstanowość: wszystkie informacje potrzebne to wykonania żądania zawarte są w zapytaniu -- jego wykonanie jest niezależne od poprzedniego stanu serwera.
			\item Pamięć podręczna: klient jest w stanie zapisać odpowiedź serwera w pamięci podręcznej aby ograniczać zapytania do serwera.
			\item System warstwowy: klient nie musi mieć bezpośredniego połączenia z serwerem
				-- pomiędzy może być zaimplementowany load-balancer, proxy lub inny serwis, który jest niewidoczny ani dla serwera ani klienta.
			\item Kod na żądanie (opcjonalne): Serwer jest w stanie przesłać wykonywalny kod na żądanie klienta, np. skrypty w języku JavaScript.
			\item Jednolity interfejs: określa wymagania dotyczące interfejsu między klientem a serwerem:
				\begin{itemize}
					\item zasoby są identyfikowane unikalnym URI
					\item serwer opisuje dane w postaci reprezentacji, klient modyfikuje dane poprzez przesłanie reprezentacji
					\item każda wiadomość zawiera informacje, w jaki sposób przetworzyć dane
					\item HATEOAS (ang.\ \emph{Hypermedia as the Engine of Application State})
						-- serwer przesyła linki do akcji, które mogą zostać wykonane dla danego stanu zasobu
				\end{itemize}
		\end{enumerate}

		Większość developerów pomija jednak implementowanie HATEOAS, ponieważ wymaga dużego nakładu pracy zarówno po stronie serwera jak i klienta
		i często nie jest potrzebne dla prostych zapytań typu CRUD.

		Przykładowa odpowiedź na zapytanie \verb|GET /accounts/12345|~\cite{RestMsg}:
		\begin{lstlisting}[label=lst:restRes]
			{
					"account": {
							"account_number": 12345,
							"balance": {
									"currency": "usd",
									"value": 100.00
							},
							"links": {
									"deposit": "/accounts/12345/deposit",
									"withdraw": "/accounts/12345/withdraw",
									"transfer": "/accounts/12345/transfer",
									"close": "/accounts/12345/close"
							}
					}
			}
		\end{lstlisting}
		
		W odpowiedzi zawarta jest lista \verb|links| --- są to akcje dostępne dla danego stanu konta.
		Jeśli stan konta byłby na minusie, część z tych akcji nie pojawiła by się w odpowiedzi.
		Takie działanie API wymaga jednak zaimplementowania maszyny stanów na serwerze.

	\subsection{GraphQL}
		GraphQL to język zapytań dla API oraz środowisko do wykonywania tych zapytań na istniejących danych~\cite{GQLsite}.
		Specyfikacja nie określa w jaki sposób ma być zbudowane Api ale oferuje następujące zasady projektowania\cite{GQLspec}:
		\begin{itemize}
			\item hierarchiczna struktura zapytań --- pola są zagnieżdżone, a zapytanie ma taki sam kształt jak dane które zwraca
			\item budowa GraphQL jest determinowana potrzebami aplikacji klienckiej
			\item silne typowanie --- w schemacie każde pole ma zdefiniowany typ, który jest walidowany
			\item zapytania określane przez klienta --- serwer GraphQL określa schemat wszystkich możliwych zapytań,
				a klient sam określa, które pola ma zwrócić
			\item introspekcyjny --- język GraphQL umożliwia odpytywanie serwera o jego schemat i typy
		\end{itemize}

		Język zapytań GraphQL (ang.\ \emph{Query Language}) znacznie różni się od SQL.
		Do pobierania danych zamiast \emph{SELECT} używany jest typ \emph{Query},
		natomiast \emph{INSERT}, \emph{UPDATE} oraz \emph{DELETE} zawierają się w jednym typie: \emph{Mutatation}.
		Dodatkowo zdefiniowany jest typ \emph{Subscription}, który używany jest do nasłuchiwania zmian przez gniazda (ang.\ \emph{socket}).
		Przykładowe zapytanie GraphQL pokazano poniżej:

		\begin{lstlisting}[label=lst:gqlQuery]
			query getAlbum($id: ID!) {
				album(id: "7803c5ee938249daa7ed67574c13e389") {
					id
					title
					albumArtist {
						name
					}
				}
			}
		\end{lstlisting}
		
		Odpowiedź z serwera na takie zapytanie ma analogiczną strukturę:

		\begin{lstlisting}[label=lst:gqlQueryRes]
			{
				"data": {
					"album": {
						"id": "7803c5ee938249daa7ed67574c13e389",
						"title": "Until the Quiet Comes",
						"albumArtist": {
							"name": "Flying Lotus"
						}
					}
				}
			}
		\end{lstlisting}

		Aby wykonanie takiego zapytania było możliwe, serwer musi zadeklarować odpowiednie typy.
		Typ reprezentuje obiekt odpowiadający jakiejś funkcji aplikacji~\cite{GraphQL}.
		W kontekście strony muzycznej do typami byłyby między innymi albumy czy artyści.
		Typ składa się z pól, które odpowiadają danym obiektu.
		Każde pole zwraca określony typ danych.

		Typy mogą być skalarne lub obiektowe: typ skalarny zawsze zwraca wartość, natomiast typ obiektowy złożony jest z kolejnych pól.
		Wbudowane typy skalarne to Int, Float, String, Boolean, ID.
		Możliwe jest tworzenie dodatkowych typów skalarnych np. DateTime, aby zapewnić odpowiednie formatowanie i walidację danych.
		Dodatkowo typy mogą zostać oznaczone wykrzyknikiem, co znaczy że zawsze zwrócą wartość (ang.\ \emph{non-nullable}).
		
		Przykładowy schemat, który umożliwi wykonanie powyższego zapytania wygląda następująco:
		
		\begin{lstlisting}[label=lst:gqlSchema]
			type Album {
				id: ID!
				title: String!
				albumArtist: Artist!
				releaseDate: DateTime
				averageRating: Float
			}

			type Artist {
				id: ID!
				albums: [Album!]!
				name: String!
			}

			type Query {
				album(id: ID!): Album
				artist(id: ID!): Artist
			}
		\end{lstlisting}

		Wszystkie zapytania wywoływane są na jednym punkcie dostępowym (np. \verb|/api/graphql|),
		przez co nie jest możliwe proste zapisywanie odpowiedzi HTTP serwera.
		Wymusza to implementację bardziej zaawansowanych mechanizmów cache po stronie klienta,
		np. poprzez identyfikacje częściowych obiektów po ID i uzupełnianie ich danych w pamięci podręcznej.

	\subsection{Porównanie}
		Żaden z tych protokołów nie jest najlepszy pod każdym aspektem i wszystkie wciąż mają swoje zastosowania.

		RPC jest najczęściej używane w API nastawionych na akcje.
		W sytuacjach, gdzie często wywołanie metody nie wpływa na stan zasobu RPC jest wystarczające.

		SOAP rozbudowuje RPC zapewniając bezpieczeństwo i transakcje.
		Istnieją jednak nowsze rozwiązania takie jak OData bazujące na REST, które mają znacznie lepszą wydajność i kompatybilność.

		REST jest najbardziej zaawansowaną architekturą (jeśli zaimplementowana w pełni) i jednocześnie najczęściej spotykaną.
		Pozwala na szybkie operacje typu CRUD oraz zaawansowaną nawigację przez HATEOAS.
		Dodatkowo dzięki unikalnym URI dla każdego zasobu umożliwia łatwe stworzenie skutecznego cache.

		GraphQL umożliwia stworzenie uniwersalnego API dla wielu klientów zapewniając wysoką wydajność.
		Definiuje jednoznaczną specyfikację zapewniającą silne typowanie oraz introspekcję w przeciwieństwie do REST,
		do którego powstało mnóstwo różniących się implementacji.

\section{Platformy do tworzenia SPA}
	Powstało mnóstwo bibliotek i frameworków do tworzenia stron internetowych i choć różnią się implementacją, to zasada działania pozostaje podobna.
	Oto trzy najpopularniejsze obecnie platformy:
	\begin{description}
		\item[Angular] to framework stworzony przez Google, napisany w języku TypeScript.
			Zawiera w sobie pełen zestaw narzędzi i bibliotek do stworzenia strony wspierającej również urządzenia mobilne.
			Definiuje wzorce projektowe, które określają strukturę projektu i dobre praktyki.

		\item[React] to biblioteka do tworzenia UI opracowana przez Facebooka.
			Do stworzenia pełnej, zaawansowanej strony wymaga użycia dodatkowych bibliotek do zarządzania stanem, trasowania czy interakcji z API.
			Nie narzuca żadnej struktury projektu, przez co każdy projekt może być zbudowany inaczej.
			Wymaga więc osoby doświadczonej, która kieruje projektem.

		\item[Vue] nie jest zarządzany przez jedną korporację, lecz przez społeczność.
			Jest najbardziej przystępny, pozwala na stworzenie zarówno prostych stron, jak i złożonych poprzez rozszerzanie infrastruktury.
			Jest to osiągalne dzięki wysoce modularnej i elastycznej strukturze.
	\end{description}

\section{Platformy do tworzenia API}
	Serwer Api można stworzyć w większości istniejących języków programowania.
	Najczęściej wykorzystywane są frameworki zbudowane w PHP (Laravel), Javie (Spring), Pytonie (Django), C\# (ASP.Net Core), Ruby (Rails) i Node.js (Express).
	W tym projekcie zdecydowano się użyć ASP.Net Core 3.0 ze względu na wieloplatformowość (wsparcie dla kontenerów linuksowych),
	wbudowany system wstrzykiwania zależności (ang.\ \emph{Dependency Injection}), nowoczesny język C\#8 oraz wysoki stopień zaawansowania bibliotek implementujących GraphQL.
	Ważnym aspektem była również oficjalna biblioteka \emph{Identity}, która zapewnia bezpieczną obsługę kont użytkowników.

\section{Docker}
	Docker dostarcza narzędzi do zarządzania zaawansowanymi technologiami kernela linuksowego, między innymi:
	LXC (ang.\ \emph{LinuX Containers}), cgroups (ang.\ \emph{Control Groups})  i kopiowanie przy zapisie (ang.\ \emph{Copy-on-write})~\cite{docker}.
	LXC wykorzystuje metody na poziomie kernela do izolowania użytkowników, procesów i sieci.
	Cgroups implementują zarządzanie zasobami oraz pomiary wykorzystania tych zasobów przez poszczególne procesy wewnątrz kontenerów.
	Pozwalają na ograniczanie i rozdzielanie dostępu do pamięci, dysku i I/O.
	Ostatnią ważną częścią Dockera jest system plików copy-on-write.
	Dzięki temu kontenery wykorzystują wspólną część plików, a dopiero po zapisie tworzona jest kopia pliku unikalna dla danego kontenera.
	Techniki te pozwoliły na uzyskanie kontenerów, które zachowują się jak maszyny wirtualne przy znacznie lepszej wydajności.
	W przeciwieństwie do maszyn wirtualnych, kontenery wykorzystują tylko tyle zasobów, ile w danej chwili potrzebują.
