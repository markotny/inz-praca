% !TEX root = ./Dyplom.tex

\chapter{Analiza wymagań projektowych}

\section{Przegląd istniejących rozwiązań}
	Na rynku obecnych jest wiele stron oferujących podobną funkcjonalność:
	\begin{itemize}
		\item Rate Your Music
		\item AllMusic
		\item Discogs
	\end{itemize}

\section{Wymagania funkcjonalne}
	\begin{enumerate}
		\item System zawiera katalog albumów
		\item Każdy album można oceniać oraz dodawać recenzję
		\item Istnieje możliwość dodawania nowych albumów
		\item Wyszukiwanie albumów korzysta z bazy zewnętrznego serwisu
		\item Możliwość rejestracji i logowania
	\end{enumerate}

\section{Wymagania niefunkcjonalne}
	\begin{enumerate}
		\item Aplikacja powinna być obsługiwana przez obecne wersje przeglądarek
		\item 
	\end{enumerate}

\section{Założenia projektowe}
	Całość aplikacji zaprojektowana zostanie ze wsparciem platformy Docker.
	Każda odrębna część systemu zamknięta będzie we własnym wirtualnym kontenerze:
	\begin{enumerate}
		\item strona internetowa typu Single-Page Application:\\
			Architektura SPA pozwala tworzyć strony, które w swoim działaniu bardziej przypominają tradycyjne aplikacje komputerowe.
			Podczas interakcji użytkownika ze stroną fragmenty widoku są dynamicznie odświeżane zamiast przeładowywania całej strony.
			Dodatkowo strona powinna przechowywać w pamięci zapytania do serwera aby minimalizować czas oczekiwania na dane.

		\item serwer uwierzytelniający użytkowników:\\
			Strona internetowa działa po stronie klienta, przez co możliwa jest znaczna ingerencja w dane.
			Aby minimalizować zagrożenia, zaimplementowane zostanie uwierzytelnianie użytkowników wykorzystujące bezpieczny protokół.
			Wykorzystany standard powinien zapewnić zarówno uwierzytelnianie jak i autoryzację.

		\item serwer dostępowy do danych aplikacji:\\
			Grafowy dostęp do bazy danych zaimplementowany zostanie za pomocą technologii GraphQL.
			Serwer umożliwiać będzie pobieranie danych stosując zapytania w języku GraphQL\@.
			Dzięki temu relacje między danymi przedstawione są w postaci grafu,
			co pozwala formować skompilowane zapytania bez potrzeby tworzenia dedykowanych kontrolerów i modeli DTO po stronie API\@.

		\item baza danych:\\
			Baza danych powinna umożliwiać przechowywanie relacji między obiektami w taki sposób,
			aby możliwe było reprezentowanie danych w postaci grafu.

		\item serwer dostarczający odwrócone proxy:\\
			Z racji tego, że użytkownicy będą przekierowywani między główną stroną, a stroną do uwierzytelniania,
			adresy serwerów zarejestrowane będą w odwróconym proxy.
			Będzie to również główny punkt dostępu do aplikacji, który będzie kierował ruchem.
			Dodatkowo, będzie obsługiwał szyfrowanie przesyłanych danych protokołem HTTPS\@.

		\end{enumerate}
		