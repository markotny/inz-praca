% !TEX root = ./Dyplom.tex

\chapter{Projekt aplikacji}
\section{Założenia projektowe}
	Całość aplikacji zaprojektowana zostanie ze wsparciem platformy Docker.
	Każda odrębna część systemu zamknięta będzie we własnym wirtualnym kontenerze:
	\begin{enumerate}
		\item strona internetowa typu Single-Page Application:\\
			Architektura SPA pozwala tworzyć strony, które w swoim działaniu bardziej przypominają tradycyjne aplikacje komputerowe.
			Podczas interakcji użytkownika ze stroną fragmenty widoku są dynamicznie odświeżane zamiast przeładowywania całej strony.
			Dodatkowo strona powinna przechowywać w pamięci zapytania do serwera aby minimalizować czas oczekiwania na dane.

		\item serwer uwierzytelniający użytkowników:\\
			Strona internetowa działa po stronie klienta, przez co możliwa jest znaczna ingerencja w dane.
			Aby minimalizować zagrożenia, zaimplementowane zostanie uwierzytelnianie użytkowników wykorzystujące bezpieczny protokół.
			Wykorzystany standard powinien zapewnić zarówno uwierzytelnianie jak i autoryzację.

		\item serwer dostępowy do danych aplikacji:\\
			Serwer umożliwiać będzie pobieranie danych stosując zapytania w języku GraphQL\@.
			Dzięki temu relacje między danymi przedstawione są w postaci grafu,
			co pozwala formować skompilowane zapytania bez potrzeby tworzenia dedykowanych kontrolerów i modeli DTO po stronie API\@.

		\item baza danych:\\
			Baza danych powinna umożliwiać przechowywanie relacji między obiektami w taki sposób,
			aby możliwe było reprezentowanie danych w postaci grafu.

		\item serwer dostarczający odwrócone proxy:\\
			Z racji tego, że użytkownicy będą przekierowywani między główną stroną, a stroną do uwierzytelniania,
			adresy serwerów zarejestrowane będą w odwróconym proxy.
			Będzie to również główny punkt dostępu do aplikacji, który będzie kierował ruchem.
			Dodatkowo, będzie obsługiwał szyfrowanie przesyłanych danych protokołem HTTPS\@.

		\end{enumerate}
		
\section{Koncepcja architektury aplikacji}
	\subsection{Uwierzytelnianie}
		Jako protokół uwierzytelniania wybrano OpenID Connect.
		Standard ten rozszerza OAuth2 (służący do autoryzacji) o warstwę identyfikacji użytkowników.
		Jest obecnie jednym z najbezpieczniejszych standardów uwierzytelniania.
		Zaimplementowany zostanie przy pomocy biblioteki \emph{IdentityServer4} na platformie \emph{ASP.Net Core 3.0}.
		Użytkownik po zalogowaniu otrzyma token JWT, który będzie zawierał cyfrową sygnaturę.
		Dzięki temu jakakolwiek ingerencja w jego strukturę sprawi, iż nie przejdzie on walidacji.

		% OBRAZEK PKCE FLOW

	\subsection{API}
		Projekt podzielony zostanie na trzy warstwy:
		\begin{description}
			\item[Api] warstwa obsługująca zapytania GraphQL
			\item[Core] warstwa domenowa definiująca modele encji oraz interfejsy, z których korzysta warstwa Api
			\item[Infrastructure] warstwa implementująca interfejsy oraz obsługująca dostęp do bazy danych
		\end{description}
		Podział ten znany jest jako ,,czysta architektura'' (ang.\ \emph{Clean architecture}).
		Taki układ umożliwia podział projektu na warstwy, które mają zdefiniowane odpowiedzialności.
		Dzięki zachowaniu zasady odwrócenia zależności możliwe jest testowanie każdej funkcjonalności z osobna.

		Na platformie \@.Net Core dostępne są dwie aktywnie rozwijane biblioteki implementujące GraphQL\@: \emph{GraphQL \@.NET} oraz \emph{Hot Chocolate}.
		Mimo znacznie mniejszej popularności wybrana została biblioteka \emph{Hot Chocolate},
		ponieważ prezentuje wiele możliwości automatyzacji generowania schematu GraphQL\@.


	\subsection{SPA}
		Do stworzenia strony zdecydowano się użyć frameworku Angular 8.
		Serwowana będzie ze środowiska uruchomieniowego Node.js.
		Projekt zbudowany jest z osobnych modułów dla każdej funkcjonalności.
		Widoki składane są z komponentów --- są to elementy zawierające szablon HTML, style CSS oraz logikę i dane w klasie napisanej w języku typescript.
		Dodatkowo komponenty mogą korzystać z serwisów. Serwisy to specjalne klasy, które są wstrzykiwane jako zależności (ang \emph{Dependency injection}).
		Zawierają się w nich dane oraz logika dzielone między wieloma komponentami.
		Nawigacja pomiędzy poszczególnymi widokami odbywa się przy pomocy dedykowanego routera, który przechwytuje nawigację przeglądarki.
		Dzięki temu nawet nawigacja do tyłu nie powoduje przeładowania całej strony tylko poszczególnych zmienionych elementów.

		Do budowy widoków użyta zostanie biblioteka Angular Material,
		która zawiera często wykorzystywane elementy zbudowane zgodnie z oficjalną specyfikacją \emph{Material design}.

		Taka modularna architektura pozwala na wielokrotne używanie tych samych komponentów oraz na zachowanie czytelnej struktury kodu.
		
Szablon składa się z pliku głównego, plików z kodem kolejnych rozdziałów i dodatków (włączanych do kompilacji w dokumencie głównym), katalogów z plikami grafik (włączanymi do rysunków w rozdziałach), pliku ze skrótami (opcjonalny), pliku z danymi bibliograficznymi (plik \texttt{dokumentacja.bib}). Taki ,,układ'' zapewnia porządek oraz pozwala na selektywną kompilację rozdziałów. Wyjaśniając to dokładniej, podczas tworzenia szablonu przyjęto następującą konwencję:
\begin{itemize}
\item Plikiem głównym jest plik \texttt{Dyplom.tex}. To w nim znajdują się deklaracje wszystkich używanych styli, definicje makr oraz ustawień, jak również polecenie \verb+\begin{document}+.
\item Teksty rozdziałów są redagowane w osobnych plikach o nazwach zawierających numer rozdziału. Pliki te zamieszczone są w katalogu głównym (tym samym, co plik \texttt{Dyplom.tex}). I~tak \texttt{rozdzial01.tex} to plik pierwszego rozdziału (ze Wstępem), \texttt{rozdzial02.tex} to plik z treścią drugiego rozdziału itd. 
\item Teksty dodatków są redagowany w osobnych plikach o nazwach zawierających literę dodatku. Pliki te, podobnie do plików z tekstem rozdziałów, zamieszczane są w katalogu głównym. I~tak \texttt{dodatekA.tex} oraz \texttt{dodatekB.tex} to, odpowiednio, pliki z treścią dodatku A oraz dodatku B.
\item Pewnym wyjątkiem od reguły nazewniczej w przypadku plików z tekstem rozdziałów i dodatków jest plik \texttt{skroty.tex}. Jest to plik, w którym zamieszczono wykaz użytych skrótów. W jego nazwie nie występuje żaden numer czy porządkowa litera. 
\item Każdemu rozdziałowi i dodatkowi towarzyszy katalog przeznaczony do składowania dołączanych w nim grafik. I tak \texttt{rys01} to katalog na pliki z grafikami dołączanymi do rozdziału pierwszego, \texttt{rys02} to katalog na pliki z grafikami dołączanymi do rozdziału drugiego itd.
Podobnie \texttt{rysA} to katalog na pliki z grafikami dołączanymi w dodatku A itd.
\item W katalogu głównym zamieszczany jest plik \texttt{dokumentacja.bib} zawierający bazę danych bibliograficznych.
\end{itemize}

\begin{table}[htb]
\centering\small
\caption{Pliki źródłowe szablonu oraz wyniki kompilacji}
\label{tab:szablon}
\begin{tabularx}{\linewidth}{|p{.55\linewidth}|X|}\hline
Źródła & Wyniki kompilacji \\ \hline\hline
\verb?Dokument.tex? - dokument główny\newline
\verb?Dokument.tcp? -- szablon projektu \texttt{MiKTeX}\newline
\verb?rozdzial01.tex? -- plik rozdziału \texttt{01}\newline
\verb?...?\newline
\verb?dodatekA.tex? -- plik dodatku \texttt{A}\newline
\verb?...?\newline
\verb?rys01? -- katalog na rysunki do rozdziału \texttt{01}\newline
\verb?   |- fig01.png? -- plik grafiki\newline
\verb?   |- ...?\newline
\verb?...?\newline
\verb?rysA? -- katalog na rysunki do dodatku \texttt{A}\newline
\verb?   |- fig01.png? -- plik grafiki\newline
\verb?   |- ...?\newline
\verb?...?\newline
\verb?dokumentacja.bib? -- plik danych bibliograficznych\newline
\verb?Dyplom.ist? -- plik ze stylem indeksu\newline
\verb?by-nc-sa.png? -- plik z ikonami CC\newline
 &
\verb?Dyplom.bbl?\newline
\verb?Dyplom.blg?\newline
\verb?Dyplom.ind?\newline
\verb?Dyplom.idx?\newline
\verb?Dyplom.lof?\newline
\verb?Dyplom.log?\newline
\verb?Dyplom.lot?\newline
\verb?Dyplom.out?\newline
\verb?Dyplom.pdf? -- dokument wynikowy\newline
\verb?Dyplom.syntex?\newline
\verb?Dyplom.toc?\newline
\verb?Dyplom.tps?\newline
\verb?*.aux?\newline 
\verb?Dyplom.synctex?\newline\\
\hline
\end{tabularx}
\end{table}

Szablon przygotowano w systemie Windows stosując kodowanie \texttt{cp1250}. Można go wykorzystać również w innych systemach i przy innych kodowaniach. Jednakże wtedy konieczna jest korekta dokumentu \texttt{Dyplom.tex} odpowiednio do wybranego przypadku. Korekta ta polegać może na zamianie polecenia \verb+\usepackage[cp1250]{inputenc}+  na polecenie \verb+\usepackage[utf8]{inputenc}+ oraz konwersji kodowania istniejących plików ze źródłem latexowego kodu (plików o rozszerzeniu \texttt{*.tex} oraz \texttt{*.bib}).

Samo kodowanie plików może być źródłem paru problemów. Chodzi o to, że użytkownicy pracujący z edytorami tekstów pod linuxem mogą generować pliki zakodowane w UTF-8 bez BOM lub z BOM, a pod windowsem -- pliki z kodowaniem znaków \texttt{cp1250} zakodowanych w~ANSI. A z takimi plikami różne edytory różnie sobie radzą (w szczególności edytor TeXnicCenter czasami z niewiadomego powodu traktuje zawartość pliku jako UTF8 lub ANSI -- chyba sprawdza, czy w bufore nie ma jakichś znaków specjalnych i na tej podstawie interpretuje kodowanie). Bywa, że choć wszystko wygląda OK to jednak kompilacja latexowa ,,nie idzie''. Problemem mogą być właśnie pierwsze bajty, których nie widać w edytorze. 

Kodowanie znaków jest istotne również przy edytowaniu bazy danych bibliograficznych (pliku \texttt{dokumentacja.bib}). Aby \texttt{bibtex} poprawnie interpretował polskie znaki plik \texttt{dokumentacja.bib} powinien być zakodowany w ANSI, CR+LF (dla ustawień jak w szablonie). Do konwersji kodowania można użyć Notepad++ (jest tam opcja ,,konwertuj'' - nie mylić z opcją ,,koduj'', która przekodowuje znaki, jednak nie zmienia sposobu kodowania pliku).



\section{Kompilacja szablonu}
Kompilację szablonu może uruchamić na killka różnych sposobów. Wszystko zależy od używanego systemu operacyjnego, zaintalowanej na nim dystrybucji latexa oraz dostępnych narzędzi. Zazwyczaj kompilację rozpoczyna się wydając polecenie z linii komend lub uruchamia się ją za pomocą narzędzi zintegrowanych środowisk.

Kompilacja z linii komend polega na uruchomieniu w katalogu, w którym rozpakowano źródła szablonu, następującego polecenia:
\begin{lstlisting}[basicstyle=\ttfamily]
> pdflatex Dyplom.tex
\end{lstlisting}
gdzie \texttt{pdflatex} to nazwa kompilatora, zaś \texttt{Dyplom.tex} to nazwa głównego pliku redagowanej pracy. 
W przypadku korzystania ze środowiska \texttt{TeXnicCenter} należy otworzyć dostarczony w szablonie plik projektu \texttt{Dyplom.tcp}, a następnie uruchomić kompilację narzędziami dostępnymi w pasku narzędziowym.

Aby poprawnie wygenerowały się wszystkie referencje (spis treści, odwołania do tabel, rysunków, pozycji literaturowych, równań itd.) kompilację \texttt{pdflatex} należy wykonać dwukrotnie, a~czasem nawet trzykrotnie, gdy wygenerowane mają zostać odwołania do pozycji literaturowych oraz wykazu literatury. 

Wygenerowanie danych bibliograficznych zapewnia kompilacja \texttt{bibtex} uruchamiana po kompilacji \texttt{pdfltex}. Można to zrobić z linii komend:
\begin{lstlisting}[basicstyle=\ttfamily]
> bibtex Dyplom
\end{lstlisting}
lub wybierając odpowiednią pozycję z paska narzędziowego wykorzystywanego środowiska. Po kompilacji \texttt{bibtex} na dysku pojawi się plik \texttt{Dyplom.bbl}. Dopiero po kolejnych dwóch kompilacjach \texttt{pdflatex} dane z tego pliku pojawią się w wygenerowanym dokumencie. Podsumowując, po każdym wstawieniu nowego cytowania w kodzie dokumentu uzyskanie poprawnego formatowania dokumentu wynikowego wymaga powtórzenia następującej sekwencji kroków kompilacji:
\begin{lstlisting}[basicstyle=\ttfamily]
> pdflatex Document.tex
> bibtex Document
> latex Document.tex
> latex Document.tex
\end{lstlisting}
Szczegóły dotyczące przygotowania danych bibliograficznych oraz zastosowania cytowań przedstawiono w podrozdziale \ref{sec:literatura}.

W głównym pliku zamieszczono polecenia pozwalające sterować procesem kompilacji poprzez włączanie bądź wyłączanie kodu źródłowego poszczególnych rozdziałów. Włączanie kodu do kompilacji zapewniają instrukcje \verb+\include+ oraz \verb+\includeonly+. Pierwsza z nich pozwala włączyć do kompilacji kod wskazanego pliku (np.\ kodu źródłowego pierwszego rozdziału \verb+\chapter{Wprowadzenie}
\label{sec:wprowadzenie}

\section{Cel pracy}
	Celem pracy była implementacja strony internetowej pozwalającej dodawać recenzje albumów muzycznych,
	wykorzystując najnowsze technologie z tej dziedziny zgodne z obecnymi trendami.
	Główny nacisk nałożony zostanie na stworzenie architektury stanowiącej solidną bazę do dalszego rozwijania aplikacji.
	Projekt wykorzystywać będzie tylko oprogramowanie open source.

\section{Zakres pracy}
	W pierwszym rozdziale przedstawiono cel oraz zakres pracy.
	Kolejny rozdział zawiera informacje na temat technologii, które zostały wykorzystanie w projekcie.
	Porównane zostały one do alternatywnych rozwiązań, które również spełniają wymagania projektu.
	Rozdział 3 opisuje wymagania projektowe: wymagania funkcjonalne i niefunkcjonalne, przypadki użycia oraz założenia projektowe.
	W następnym rozdziale przedstawiono projekt aplikacji.
	Zawarte w nim są projekty architektury poszczególnych warstw aplikacji oraz bazy danych.
	Rozdział 5 zawiera opis implementacji projektu.
	\todo{opisać strukture pracy jak będą wszystkie rozdziały}
+). Druga, jeśli zostanie zastosowana, pozwala określić, które z~plików zostaną skompilowane w całości (na przykład kod źródłowy pierwszego i drugiego rozdziału \verb+\includeonly{rozdzial01.tex,rozdzial02.tex}+).
Brak nazwy pliku na liście w poleceniu \verb+\includeonly+ przy jednoczesnym wystąpieniu jego nazwy w poleceniu \verb+\include+ oznacza, że w kompilacji zostaną uwzględnione referencje wygenerowane dla tego pliku wcześniej, sam zaś kod źródłowy pliku nie będzie kompilowany. 

W szablonie wykorzystano klasę dokumentu \texttt{memoir} oraz wybrane pakiety. Podczas kompilacji szablonu w \texttt{MikTeXu} wszelkie potrzebne pakiety zostaną zainstalowane automatycznie (jeśli \texttt{MikTeX} zainstalowano z opcją dynamicznej instalacji brakujących pakietów). W przypadku innych dystrybucji latexowych może okazać się, że pakiety te trzeba doinstalować ręcznie (np.\ pod linuxem z \texttt{TeXLive} trzeba doinstalować dodatkową zbiorczą paczkę, a jeśli ma się menadżera pakietów latexowych - to pakiety latexowe można instalować indywidualnie).

Jeśli w szablonie będzie wykorzystany indeks rzeczowy, kompilację źródeł trzeba będzie rozszerzyć o kroki potrzebne na wygenerowanie plików pośrednich \texttt{Dokument.idx} oraz \texttt{Dokument.ind} oraz dołączenia ich do finalnego dokumentu (podobnie jak to ma miejsce przy generowaniu wykazu literatury).
Szczegóły dotyczące generowania indeksu rzeczowego opisano w podrozdziale~\ref{sec:indeks}.