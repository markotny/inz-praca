\chapter{Wprowadzenie}
\section{Cel i zakres pracy}
	Celem pracy jest implementacja strony internetowej pozwalającej dodawać recenzje albumów muzycznych,
	wykorzystując najnowsze technologie z tej dziedziny zgodne z obecnymi trendami.

	Główny nacisk nałożony zostanie na stworzenie architektury stanowiącej solidną bazę do dalszego rozwijania aplikacji.
	Projekt wykorzystywać będzie tylko oprogramowanie open source.

	Zakres pracy obejmuje:
	\begin{itemize}
		\item analiza 
		\item opracowanie struktury aplikacji
		\item implementacja aplikacji
		\item testy 
	\end{itemize}
	
\section{Układ pracy}

\section{Przegląd technologii}
	\subsection{Paradygmaty komunikacji z usługami sieciowymi}
		\begin{description}
			\item[SOAP] to ustandaryzowany protokół służący do opakowywania przesyłanych danych. 
				Często porównywany do wkładania danych do ,,koperty'', definiuje sposób kodowania i dekodowania danych do formatu XML \cite{SOAP}.
				Paczka z danymi zawiera również informacje opisujące dane oraz w jaki sposób mają być przetworzone.
				Wykorzystuje przez to więcej zasobów niż czysty JSON z danymi -- zarówno w kwestii rozmiaru przesyłanych danych jak i czasu ich przetworzenia.
				Zapytanie w SOAP jest zawsze typu POST, a więc odpowiedź nie jest zapisywana przez przeglądarkę i za każdym razem wywoływana jest procedura w Api.
				Jest jednak dzięki temu bardziej zaawansowany - zapewnia wsparcie dla transakcji spełniających ACID
				oraz posiada rozszerzone mechanizmy bezpieczeństwa -- oprócz SSL, implementuje również WS-Security.
				Te właściwości są istotnymi zaletami wykorzystywanymi w aplikacjach typu enterprise.

				% budowa SOAP Message (envelope)

			\item[REST] w przeciwieństwie do \emph{SOAP} oraz \emph{GraphQL}, \emph{REST} nie jest protokołem lecz stylem architektury.
				Nie narzuca konkretnej implementacji ani specyfikacji. Definiuje za to wymagania, które powinno spełnić Api typu RESTful \cite{RESTful}:
				\begin{enumerate}
					\item Klient-Serwer: podział ról i odpowiedzialności -- klient i serwer mogą być rozwijane niezależnie, o ile interfejs pozostanie bez zmian.
					\item Bezstanowość: wszystkie informacje potrzebne to wykonania żądania zawarte są w zapytaniu -- jego wykonanie jest niezależne od poprzedniego stanu serwera.
					\item Pamięć podręczna: klient jest w stanie zapisać odpowiedź serwera w pamięci podręcznej aby ograniczać zapytania do serwera.
					\item System warstwowy: klient nie musi mieć bezpośredniego połączenia z serwerem
						-- pomiędzy może być zaimplementowany load-balancer, proxy lub inny serwis, który jest niewidoczny ani dla serwera ani klienta.
					\item Kod na żądanie (opcjonalne): Serwer jest w stanie przesłać wykonywalny kod na żądanie klienta, np. skrypty w języku JavaScript.
					\item Jednolity interfejs: określa wymagania dotyczące interfejsu między klientem a serwerem:
						\begin{itemize}
							\item zasoby są identyfikowane unikalnym URI
							\item serwer opisuje dane w postaci reprezentacji, klient modyfikuje dane poprzez przesłanie reprezentacji
							\item każda wiadomość zawiera informacje, w jaki sposób przetworzyć dane
							\item HATEOAS (ang.\ \emph{Hypermedia as the Engine of Application State})
								-- serwer przesyła linki do akcji, które mogą zostać wykonane dla danego stanu zasobu
						\end{itemize}
				\end{enumerate}

				Z powodu braku jednolitej specyfikacji, istnieje wiele różnych implementacji tych wymagań.
				Często są one zaniedbywane, a Api nazywane REST-owym tak naprawdę nie różni się wtedy za wiele od RPC (ang.\ \emph{Remote Procedure Call}).
				Z tego powodu zaczęto określać stopień zaawansowania implementacji (rys X) --- większość implementacji 

				% OBRAZEK Z https://www.martinfowler.com/articles/richardsonMaturityModel.html



			\item[GraphQL] 
		\end{description}

	\subsection{Platformy do tworzenia SPA}
		\begin{description}
			\item[Angular]
			\item[React] 
			\item[Vue] 
		\end{description}

	\subsection{Platformy do tworzenia API}
		\begin{description}
			\item[ASP.Net Core]
			\item[Node.js] 
			\item[Django] 
		\end{description}