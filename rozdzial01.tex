\chapter{Wstęp}
\section{Wprowadzenie}
Niniejszy dokument powstał z myślą o ujednoliceniu sposobu redagowania prac dyplomowych. Jego źródła mają pełnić rolę szablonu nowoedytowanej pracy, zaś treść powinna być interpretowana jako zestaw zaleceń i uwag o charakterze technicznym (dotyczących takich zagadnień, jak: formatowanie tekstu, załączanie rysunków, układ strony itp.) oraz stylistycznym (odnoszących się do stylu wypowiedzi, sposobów tworzenia referencji itp.).

Szablon przygotowano do kompilacji pdflatexem w konfiguracji: \texttt{MiKTeX} (windowsowa dystrybucja latexa) + \texttt{TeXnicCenter} (środowisko do edycji i kompilacji projektów latexowych) + \texttt{SumatraPDF} (przeglądarka pdfów z nawigacją zwrotną) + JabRef (opcjonalny edytor bazy danych bibliograficznych). Jest to zalecany zestaw narzędzi do edycji pracy w systemie Windows. Można je pobrać ze stron internetowych, których adresy zamieszczono w tabeli~\ref{tab:narzedzia}.
\begin{table}[htb] \small
\centering
\caption{Wykaz zalecanych narzędzi do kompilacji szablonu (adresy internetowe ważne na dzień 1.04.2016)}
\label{tab:narzedzia}
\begin{tabularx}{\linewidth}{|c|c|X|p{6cm}|} \hline\
Narzędzie & Wersja & Opis & Adres \\ \hline\hline
MiKTeX & 2.9 & Zalecana jest instalacja \texttt{Basic MiKTeX} z dystrubucji 32 lub 64 bitowej. Brakujące pakiety będą się doinstalowywać podczas kompilacji projektu. &
\url{http://miktex.org/download} \\ \hline
TexnicCenter & 2.02 &  Można pobrać 32 lub 64 bitową wersję & \url{http://www.texniccenter.org/download/} \\ \hline
SumatraPDF & 3.1.1 & Można pobrać 32 lub 64 bitową wersję & \url{http://www.sumatrapdfreader.org/download-free-pdf-viewer.html} \\ \hline
JabRef & 3.3 & Można pobrać 32 lub 64 bitową wersję & \url{http://www.fosshub.com/JabRef.html} \\ \hline
\end{tabularx}
\end{table}
Nic nie stoi jednak na przeszkodzie, aby szablon ten dostosować do wykorzystania z użyciem innych narzędzi. Dostosowanie to mogłoby polegać na zmianie kodowania plików i korekcie deklaracji kodowania znaków w dokumencie głównym (co opisano dalej).

W kodzie źródłowym szablonu zamieszczono komentarze z uwagami pozwalającymi lepiej zrozumieć znaczenie używanych komend. Komentarze te nie są widoczne w pliku \texttt{Dokument.pdf}, który powstaje jako wynik kompilacji szablonu. 


%%%2. środowisko do pisania kodu latexa: 
%%%( )
%%%3. viewer pdf-ów, pozwalający na nawigację zwrotną: Sumatra PDF 3.0
%%%(http://www.sumatrapdfreader.org/download-free-pdf-viewer.html)
%%%
%%%- o konfiguracji texniccenter do współdziałania z sumatra pdf można poczytać sobie na stronie:
%%%http://tex.stackexchange.com/questions/116981/how-to-configure-texniccenter-2-0-with-sumatra-2013-2014-2015-version
%%%(można znaleźć też inne tutoriale)
%%%
%%%4. środowisko do zarządzania bibliografią: JabRef
%%%(http://jabref.sourceforge.net/download.php)
%%%
%%%Polecam też instalację pod windowsami następujących narzędzi:
%%%- Sumatra PDF - przeglądarka pdf umożliwiająca nawigację pomiędzy
%%%edytowanym tekstem a przeglądanym dokumentem (podglądanie tekstu w
%%%TeXnicCenter umieszcza kursor w odpowiednim miejscu w pdfie, podwójne
%%%kliknięcie w pdfie ustawia kursor w edytorze tekstu).
%%%- JabRef - narzędzie do przygotowywania bibliografii.
%%%
%%%
%%%Uwaga: tytuł powinien zmieścić się w okienku kolorowej okładki (którą
%%%powinna dostarczyć uczelniana administracja). Proszę posterować
%%%parametrami, aby "wpasować" w okienko własny tekst.
%%%
%%%Do ASAPa należy wprowadzić pracę dyplomową/projekt inżynierski w pliku o nazwie:
%%%
%%%W04_[nr albumu]_[rok kalendarzowy]_[rodzaj pracy] (szczegółowa instrukcja pod adresem asap.pwr.edu.pl)
%%%
           %%%Przykładowo:
        %%%­W04_123456_2015_praca inżynierska.pdf     - praca dyplomowa inżynierska
        %%%W04_123456_2015_projekt inżynierski.pdf   - projekt inżynierski
        %%%W04_123456_2015_praca magisterska.pdf  - praca dyplomowa magisterska
%%%
              %%%rok kalendarzowy ? rok realizacji kursu „Praca dyplomowa” (nie rok obrony) 