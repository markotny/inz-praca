\chapter{Wprowadzenie}
\label{sec:wprowadzenie}

\section{Cel pracy}
	Celem pracy jest implementacja strony internetowej pozwalającej dodawać recenzje albumów muzycznych,
	wykorzystując najnowsze technologie z tej dziedziny zgodne z obecnymi trendami.
	Główny nacisk nałożony został na stworzenie architektury stanowiącej solidną bazę do dalszego rozwijania aplikacji.
	Projekt wykorzystuje tylko oprogramowanie open source.

	W ramach projektowania aplikacji szczególna uwaga zwrócona zostanie na użyteczność wykorzystania GraphQL do uzyskania grafowego dostępu do danych.
	GraphQL często uważane jest za rewolucyjne podejście do tworzenia API, co zostanie zbadane w tej pracy.
	Dużo uwagi poświęcone zostanie również kwestii bezpieczeństwa logowania do aplikacji.

\section{Zakres pracy}
	W pierwszym rozdziale przedstawiono cel oraz zakres pracy.
	Kolejny rozdział zawiera informacje na temat technologii, które zostały wykorzystanie w projekcie.
	Porównane zostały one do alternatywnych rozwiązań, które również spełniają wymagania projektu.
	Rozdział 3 opisuje wymagania projektowe: wymagania funkcjonalne i niefunkcjonalne, przypadki użycia oraz założenia projektowe.
	W następnym rozdziale przedstawiono projekt aplikacji.
	Zawarte w nim są projekty architektury poszczególnych warstw aplikacji oraz bazy danych.
	Rozdział 5 zawiera opis implementacji projektu.
	Podzielony został na podrozdziały, które opisują po kolei implementację każdego dockerowego kontenera aplikacji.
	Kolejny rozdział opisuje przeprowadzone testy aplikacji.
	W 7 rozdziale zaprezentowano zrzuty ekranu powstałej aplikacji.
	W ostatnim rozdziale zawarto podsumowanie pracy.
