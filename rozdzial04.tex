\chapter{Implementacja}
\section{Strona internetowa}
	\subsection{Zastosowane technologie}
		Zgodnie z projektem, strona stworzona została w Angularze.
		Oprócz podstawowych bibliotek Wykorzystane zostały następujące biblioteki:
		\begin{description}
			\item[flex-layout] --- biblioteka stworzona przez zespół tworzący Angulara, umożliwiająca stworzenie responsywnego interfejsu.
				Dostarcza API, które pozwala na definiowanie struktury elementów HTML zależnej od rozmiaru ekranu.
				Dzięki temu interfejs automatycznie dostosowuje się np. do ekranu telefonu komórkowego.

			\item[apollo] --- wiodący klient GraphQL. Oprócz implementacji protokołu GraphQL,
				zapewnia również zarządzanie pamięcią podręczną (ang.\ \emph{cache}) oraz stanem aplikacji (ang.\ \emph{state management}).
				Dzięki temu wszystkie wszystkie odpowiedzi z serwisu Api są zapamiętywane,
				co pozwala na stworzenie aplikacji działającej szybko nawet przy słabym połączeniu z internetem.

			\item[oidc-client] --- biblioteka implementująca protokół OpenID Connect oraz OAuth2.
				Zapewnia klasy obsługujące proces logowania oraz zarządzania tokenami.

			\item[rxjs] --- ReactiveX 
		\end{description}